\chapter{Results}
\label{ch:results}
\section{Transgenic animals}
\label{sec:transgenic animal}
This project examines the interactions between GABAergic interneurons and acetyl cholinergic axons in the barrel cortex. To identify cell types, I used transgenic mice that express fluorescent proteins in my structures of interest. Cholinergic axons are labelled with choline-acetyltransferase-driven EYFP expression in the ChAT-ChR2-EYFP mouse line (fig. \ref{fig:animal}A; \cite{Zhao2011}). Interneurons are labelled with tdTomato expression driven by VIP and PV reporters respectively (fig. \ref{fig:animal}B, C; \cite{Taniguchi2012}; \cite{Arber2015}). Through crossing of ChAT mice with either VIP (fig. \ref{fig:animal}D) or PV mice (fig. \ref{fig:animal}E), simultaneous labelling of both structures allow investigating their interaction. Unfortunately, tdTomato has a wide excitation spectrum which overlaps with EYFP, creating bleed-through in the EYFP (green) channel (fig. \ref{fig:animal}D, E). The narrower EYFP excitation range prevents bleed-through in the tdTomato channel.

\begin{figure}
	\captionsetup[figure]{indentation=0pt}
	\includegraphics[height=.8\textheight]{"animal_Nelo".pdf}
	\caption{\textbf{Illustration of the transgenic mice lines used for my study.} \textbf{(A-C)} Schematic illustration of the genetic manipulations performed to obtain ChAT-ChR2-EYFP, VIP-ires-tdTomato and PV-ires-tdTomato mice lines with their respective low magnification fluorescent images (80~µm thick brain slices at 10X magnification, scale bar = 500 $\mu m$. \textbf{(D)}. Confocal images at 20X (upper panel) and 40X (lower pannel) of a VIPcre-tdTomato/ChAT-ChR2-EYFP mouse. \textbf{(E)} Confocal images of PVcre-tdTomato/ChAT-ChR2-EYFP mouse at 20X (upper panel) and 40X images (lower panel). Scale bars = 20~$\mu m$. }	
	\label{fig:animal}
\end{figure}

\section{Morphological description of the VIP-expressing interneuron cell body and nucleus}
\label{sec:VIP cell body}
I first analysed the morphology of imaged and reconstructed VIP-expressing interneurons. Two sections of one VIPcre-tdTomato mouse could be imaged, yielding 4 reconstructed VIP+ cells (fig. \ref{fig:vip_cell_body}). Their cell bodies have an average volume of 583.064 ± 71.658~$\mu m^3$ (table \ref{tab:vip_cell_body}). Somata are largely round to slightly oval, and have very few, large dendrites (fig. \ref{fig:vip_cell_body}A). Imaged cells have bipolar or bitufted structures, which is typical for VIP+ interneurons. A striking common  characteristic of VIP+ neurons is the highly irregular shape of their nuclei, as can be seen in the EM images and in the resulting 3D reconstruction (fig. \ref{fig:vip_cell_body}B, C). The nuclei appear elongated, with complex foldings and rough textures on their membranes. All imaged VIP+ cells have this irregular nucleus.\\
\captionof{table}{\textbf{Quantification of cell body volume of four imaged and reconstructed VIP+ cells.}}
\label{tab:vip_cell_body}
\begin{tabular}{|l|l|}
	\hline
	Cell number & Cell body volume ($\mu m^3$)\\
	\hline
	VIP1 & 523.841\\
	\hline
	VIP2 & 695.598\\
	\hline
	VIP3 & 518.021\\
	\hline
	VIP4 & 594.797\\
	\hline
    VIP average & 583.064 $\pm$ 71.658\\
	\hline
\end{tabular}
\begin{figure}
	\captionsetup[figure]{indentation=0pt}
	\includegraphics[width=\linewidth]{"vip_cell_body_Nelo".pdf}
	\caption{\textbf{Morphologic description of the VIP-positive interneuron's cell body.} \textbf{(A)} Z-projection of a stack of confocal images of VIP-tdTomato/ChAT-ChR2-EYEP showing the 2 VIP+ cells and corresponding 3D EM reconstruction. Scale bar = 20~\textmu m. \textbf{(B)} 3D EM reconstruction and EM image of the left VIP+ on the left showing its nucleus and cell body features. \textbf{(C)} 3D EM models (left top and bottom) and EM image (right) of the VIP+ cell body and nucleus. EM scale bars = 5~\textmu m.}
	\label{fig:vip_cell_body}
\end{figure}
\section{Morphological analysis of VIP-expressing interneurons' dendrites}
\label{sec:VIP dendrites}
As mentioned in chapter \ref{sec:vip distribution}, VIP+ cells can have a bipolar or bitufted structure, with one or two large dendrites that show limited branching. A bitufted morphology can be observed in all imaged cells (fig. \ref{fig:vip_dendrites} A,B). As the vasointestinal peptide is a cytoplasmic marker and its distribution is not limited to the soma, VIP-positive dendrites whose somata are outside the field of view (FOV) can be imaged and reconstructed. This offers the possibility to compare proximal dendrites (cell body in the FOV) with distal dendrites (soma outside the FOV). Synaptic connections on proximal and distal dendrites were mainly found on the dendritic shaft but a few were formed via spines, small  membrane protuberances (fig. \ref{fig:vip_dendrites}C). Spines on proximal VIP+ dendrites have an average neck length of ~ 0.4~$\mu m$ and an average head volume of ~0.1~$\mu m^3$. However, there are no spines on the analyzed distal dendrites. Overall, 214~$\mu m$ of proximal and 86~$\mu m$ of distal dendrites could be reconstructed (table \ref{tab:vip_dendrites}). The analysis of synaptic contacts reveals that VIP dendrites generally have more excitatory than inhibitory synapses. However, this relation differs drastically between proximal and distal dendrites. The ratio between excitatory and inhibitory synapse frequencies (i.e. how many synapses you would on average expect on 1~$\mu m$ of dendrite) can serve as an indicator of the relative importance of excitatory and inhibitory input (see chapter \ref{sec:stat}). Whereas proximal dendrites have an excitatory input ratio of 25.5 (i.e. proximal dendrites have 25 times more excitatory than inhibitory synapses), distal dendrites have a ratio of only 5 (table \ref{tab:vip_dendrites}). GABAergic and glutamatergic synapses appear to have similar sizes (fig. \ref{fig:vip_dendrites}D), thus a similar trend is visible when considering synaptic surface instead of synapse numbers, with proximal dendrites having a ratio of 13.71 versus 5.17 for distal dendrites (table \ref{tab:vip_dendrites}). However, distal synapses appear to be smaller than contacts on proximal dendrites, although this trend is not statistically significant (P = 0.051, two-way ANOVA, F(1,167)=3.858, fig. \ref{fig:vip_dendrites}D).\\
\captionof{table}{\textbf{Analysis of proximal and distal VIP+ dendrites.} Data pooled from X proximal dendrites belonging to 4 reconstructed VIP+ neurons and Y distal dendrites belonging to an unknown number of VIP+ cells. Proximal dendrites have their soma in the field of view, whereas distal dendrites have their soma outside. Formulas for input ratios are described in chapter \ref{sec:stat}.}
\label{tab:vip_dendrites}
\begin{center}
	\begin{tabular}[c]{|l|c|r|}
		\hline
		Measurement & Proximal & Distal\\
		\hline
		N. excitatory synapses & 110 & 47\\
		\hline
		N. Inhibitory synapses & 4 & 10\\
		\hline
		Dendrite length [$\mu m$] & 214.196 & 86.266\\
		\hline
		Excitatory surface [$\mu m^2$] & 27.07 & 8.86\\
		\hline
		Inhibitory surface [$\mu m^2$] & 1.98 &  1.72\\
		\hline
		Dendrite surface [$\mu m^2$] & 564.35 & 203.90\\
		\hline
		Excitatory frequency [$\mu m ^-1$] & 0.51 & 0.55\\
		\hline
		Inhibitory frequency [$\mu m ^-1$] & 0.02 & 0.11\\
		\hline
		Excitatory input ratio (N. syn.) & 25.50 & 5.00\\
		\hline
		Excitatory surface ratio [\%] & 4.80 & 4.34\\
		\hline
		Inhibitory surface ratio [\%] & 0.35 & 0.84\\
		\hline	
		Excitatory input ratio (area) & 13.71 & 5.17\\
		\hline
	\end{tabular}
\end{center}

\begin{figure}
	\captionsetup[figure]{indentation=0pt}
	\includegraphics[width=\linewidth]{"vip_dendrites_v2_Nelo".pdf}
	\caption[Comparison of VIP+ proximal and distal dendrites.]{\textbf{Comparison of VIP+ proximal and distal dendrites.} \textbf{(A)} Representative fluorescent images at low (10X, left) and high (40X, right) magnification of VIP+ cells. Scale bars =  50 (left) and 20 (right) \textmu m. \textbf{(B)} 3D EM reconstructions of proximal (left) and distal (right) VIP+ dendrites. \textbf{(C)} Representative EM image of a proximal VIP+ spine with inhibitory (blue) and excitatory (red) synapse. Quantification of VIP+ spine head volumes and spine neck length (n = 4). \textbf{(D)} Excitatory and inhibitory synapse sizes on proximal and distal dendrites. Two-way ANOVA, F(1,167)=3.858, P=0.051. N listed in table \ref{tab:vip_dendrites}. \textbf{(E)} Relative abundance of inhibitory (red) and excitatory (blue) synapses on proximal and distal synapses.}	
	\label{fig:vip_dendrites}
\end{figure}
\section{Morphological description of the PV-expressing interneuron cell body}
\label{sec:PV cell body}
Next, I reconstructed and analysed parvalbumin-expressing interneurons. One section of one mouse expressing tdTomato in all PV+ neurons could be analysed, where one PV+ cell was imaged and reconstructed.\\
In contrast to the more regular, bipolar structure of the VIP neurons, the PV+ neuron has a rounder soma with several, highly branching dendrites extending in all directions (fig. \ref{fig:pv_cell_body}B, C). With 1555.29~$\mu m^3$, the PV+ cell is nearly three times larger than the VIP+ neurons. Its nucleus is shaped irregularly, similar to the nuclei of the VIP+ cells, with deep foldings, but a more round shape (fig. \ref{fig:pv_cell_body}D, E).\\
\begin{figure}
	\captionsetup[figure]{indentation=0pt}
	\includegraphics[width=\linewidth]{"pv_cell_body_Nelo".pdf}
	\caption[Cell body and nucleus of a PV+ interneuron.]{\textbf{Cell body and nucleus of a PV+ interneuron.} \textbf{(A)} 10x confocal overview of PV expression in S1. Square shows ROI with reconstructed cell. Scale bar = ~$\mu m$. \textbf{(B and C)} 40x confocal image of the reconstructed PV+ cell and corresponding 3D EM reconstruction. Main dendrites are marked with lines. Scale bar = 20~$\mu m$. \textbf{(D and E)} 3D EM reconstruction and EM image of PV+ soma and nucleus. Scale bar = 5~$\mu m$.}
	\label{fig:pv_cell_body}
\end{figure}
\section{Morphological analysis PV-expressing interneuron dendrites}
\label{sec:PV dendrites}
As mentioned before, multiple (n = 6) dendrites of the PV+ neuron emerge from the soma in all directions and form branches repeatedly, giving the cell a ramified shape (fig. \ref{fig:vip_cell_body}C, \ref{fig:pv_dendrites}). Overall, dendrites with a total length of \textasciitilde 327~$\mu m$ and a surface of 1799~$\mu m^2$ were analyzed. Similarly to VIP+ cells, PV+ dendrites receive more excitatory than inhibitory input (table \ref{tab:pv_dendrites}), although the ratio is not as skewed as in the VIP+ cells. Furthermore, the average synapse size is relatively small, with excitatory synapses measuring 0.15~$\mu m^2$. Inhibitory synapses are with \textasciitilde 0.08~$\mu m^2$ even smaller (P<0.001, fig. \ref{fig:pv_dendrites}F). All analyzed synaptic connections occur directly on the dendritic shaft (fig. \ref{fig:pv_dendrites}B). Only 2 membrane protrusions are observed on PV+ dendrites with spine head volumes of 0.4~$\mu m^3$ and 0.01 $\mu m^3$. Furthermore, none of the spines form synaptic contacts. In combination with the extremely small size of the second spine, it is difficult to assess whether it may be a still developing spine or a random membrane protrusion. Spine neck length is \textasciitilde 1~$\mu m$ for both spines. Unfortunately, no distal PV+ dendrites could be identified, and a comparison similar to chapter \ref{sec:VIP dendrites} is not possible. \\
\captionof{table}{\textbf{Quantification of synaptic input on PV+ dendrites.} }
\label{tab:pv_dendrites}
\begin{tabular} {|l|l|l|l|l|l|}
	\hline
	Num. ex. syn. & Num. inh. syn. & Dend. length  & Ex. freq. & Inh. freq. \\
	\hline
	309 & 42 & 327.487 $\mu m$ & 0.94 $\mu m^-1$ & 0.13 $\mu m^-1$\\
	\hline
	Ex. surface & Inh. surface & Dend. surface & Ex. ratio (\%) & Inh. ratio (\%)  \\
	\hline
	46.895 $\mu m^2$ & 9.000 $\mu m^2$ & 1799.022 $\mu m^2$ & 2.60 & 0.50 \\
	\hline
\end{tabular}
\begin{figure}
	\captionsetup[figure]{indentation=0pt}
	\includegraphics[width=\linewidth]{"pv_dendrites_v2_Nelo".pdf}
	\caption[Synaptic input into PV+ interneurons.]{\textbf{Analysis of synaptic input onto proximal PV+ interneurons.} \textbf{(A)} Complete 3D EM reconstruction of a PV+ dendrite. \textbf{(B and C)} High magnification and overview of reconstructed synapses on the dendritic shaft of a PV+ dendrite. \textbf{()D)} Reconstructions of the other 4 PV+ dendrites. \textbf{(E)} Representative EM images of excitatory (left) and inhibitory (right) synapses (scale = 1 $\mu m$). \textbf{(D)} Distribution of the glutamatergic and GABAergic synapse surface area. Unpaired Student's t-test, t=4.884, df=349, P<0.001. Normal distribution confirmed before with Kolmogorov-Smirnov test (P<0.001).}
	\label{fig:pv_dendrites}
\end{figure}
\section{Morphological analysis of ChAT-expressing axons}
\label{sec:ChAT axons}
Next, the acetyl cholinergic axons found in the region of interest were reconstructed. A total of 333.78 $\mu m$ of  7 cholinergic axons from two independent experiments were analyzed (fig. \ref{fig:chat_axons}A). These elements are relatively thin with enlargements where the vesicles accumulate, highly ramified and unmyelinated (fig. \ref{fig:chat_axons} B and C). To assess the cholinergic output, regions contacting dendritic shafts or spines with 3 or more vesicles were considered functional contacts (fig. \ref{fig:chat_axons} C). The contact surface area (\textasciitilde 0.08 $\mu m^2$) was consistent across experiments. A contact density of 0.16 synapse per $\mu m$ axon length was observed (table \ref{tab:chat axons}).\\
\captionof{table}{\textbf{Morphological quantification of ChAT+ axons in the barrel cortex.} Ratios show number of contacts per \textmu m (column 4) or relative axon surface occupied by membrane contacts (column 7). Data pooled from 7 reconstructed axons from 2 sections.}
\label{tab:chat axons}
\begin{tabular} {|l|l|l|l|l|l|l|}
	\hline
	& N. cont. & Axon length & Ratio & Cont. surf. & Axon surface & Ratio\\ 
	\hline
	ChAT-VIP & 24 & 214.22 $\mu m$ & 11.2 & 2.47 $\mu m^2$ & 370.56 $\mu m^2$ & 0.67 \%\\
	\hline
	ChAT-PV & 31 & 119.56 $\mu m$ & 25.6 & 2.21 $\mu m^2$ & 241.87 $\mu m^2$ & 0.91 \%\\
    \hline
    Total & 55 & 333.78 $\mu m$ & 16.48 & 4.68 $\mu m^2$ & 612.43 $\mu m^2$ & 0.76 \%\\
    \hline 
\end{tabular}
\begin{figure}
	\captionsetup[figure]{indentation=0pt}
	\includegraphics[width=\linewidth]{"chat_axons_Nelo".pdf}
	\caption[ChAT-positive axons]{\textbf{ChAT-positive axons.} \textbf{(A and B)} Fluorescent images of ChAT+ axons in the VIP (A) and PV (B) experiment as well as their 3D EM reconstructions (middle: overview, right: single axons). Scale bar = 20~$\mu m$. \textbf{(C)} Representative EM images of cholinergic axons apposing other structures. Scale bar = 5~\textmu m. \textbf{(D)} Quantification of the ACh contact size to VIP+ and PV+ neurons. Asterisks show P-values of Kolmogorov-Smirnov test for normal distributions.}
	\label{fig:chat_axons}
\end{figure}
\section{ChAT-expressing axons and their interaction with GABAergic interneurons}
\label{sec:GABA+ChAT}
\subsection{VIP+ interneurons and ChAT+ axons interaction}
\label{subsec:vip+chat}
First I assessed the interaction between VIP-expressing cells and cholinergic axons in layer II/III of barrel cortex (fig. \ref{fig:vip_chat}). Ultrastructural EM imaging and subsequent 3D reconstruction showed that ChAT-positive axons contact VIP+ cell body and dendrites in a synaptic and non-synaptic manner (fig. \ref{fig:vip_chat}D, E and F). These contact are observed also onto distal dendrites (fig. \ref{fig:vip_chat}C). The contact surface between the two structures of interest was quantified grouped per VIP+ cell body, proximal and distal dendrites (fig. \ref{fig:vip_chat}G). According to my measurement, the proximal dendrites have a larger absolute contact surface (\textasciitilde 0.52 $\mu m^2$) than the cell body and distal dendrites. However, when considering the smaller membrane surface of the distal dendrites, the ACh contacts cover nearly double the area on distal compared to proximal dendrites. (\textasciitilde 0.18 \%, fig. \ref{fig:vip_chat}G). 
\begin{figure}[!h]
	\captionsetup[figure]{indentation=0pt}
	\includegraphics[width=\linewidth]{"vip_chat_Nelo".pdf}
	\caption[Schematic illustration of the imaging chamber]{\textbf{VIP-positive interneurons and acetyl cholinergic axons interaction. (A)} ChAT+ axons (top), VIP+ neurons (mid) and merged confocal 40x images (bottom). Scale bar = 20~$\mu m$. \textbf{(B)} 3D EM reconstructions of VIP+ neurons and surrounding ChAT+ axons seen in A. Boxes and letters indicate contact locations shown in C-F. \textbf{C-F} 3D reconstructions and EM micrographs of contacts between ChAT+ axons and VIP+ proximal dendrites (E and F), distal dendrites (C) and soma (D). (G and H). Quantification of ChAT-VIP interaction in different regions of the neuron, as measured in total contact surface (G) and relative contact surface (H) normalized against total axonal surface.}
	\label{fig:vip_chat}
\end{figure}
\subsection{PV+ interneurons and ChAT axons interaction}
\label{subsec:PV+ChAT}
In this experiment a PV+ cell of layer II/III and the cholinergic axons in its proximity were reconstructed and analyzed (fig. \ref{fig:pv_chat}). Even though the cholinergic axon have higher synaptic output in this region, 25.6 \%, compared to 11.2 \% in the region of VIP+, (table \ref{tab:chat axons}), here only one ChAT/PV contact with ACh vesicles was found, located on the PV+ cell body (fig. \ref{fig:pv_chat}). Surprisingly, no ChAT+ contact was found within the \textasciitilde 327 $\mu m$ of PV+ dendrites (table \ref{tab:pv_dendrites}). However, the single contact found on the cell body shows no clear synapse, indicating that ChAT+ axon have a lower tendency to contact PV+ cells then VIP+ cells. 
\begin{figure}
	\captionsetup[figure]{indentation=0pt}
	\includegraphics[width=.8\linewidth]{"pv_chat_Nelo".pdf}
	\caption[PV-positive interneuron and cholinergic axon interactions.]{\textbf{PV-positive interneuron and cholinergic axon interactions.} \textbf{(A)} Representative fluorescent images of a PV+ cell (top), an adjacent ChAT+ axon (mid) and merged channels (bottom). Scale bar = 20~$\mu m$. \textbf{(B)} 3D EM reconstruction of the imaged PV+ cell and the ChAT+ axon in close proximity, indicated by the box. \textbf{(C-F)} 3D reconstruction and EM micro of ChAT+/PV+ contacts. Scale bar = 5~$\mu m$.}
	\label{fig:pv_chat}
\end{figure}
\section{VIP+ and PV+ axons characterization}
\label{sec:VIP and PV axon}
It was possible to identify and reconstruct the axon of one VIP+ and one PV+ cell. Several differences are apparent between the axons of these cell types (table \ref{tab:pv_vip_axons}). The VIP+ axon is extending relatively straight from the soma, without branching (fig. \ref{fig:vip_pv_axons}A). Additionally, the axon formed two synaptic outputs in the reconstructed region (fig. \ref{fig:vip_pv_axons} Ac, Ad). At the place of the contact, the axon increases in diameter and accumulates vesicles (fig. \ref{fig:vip_pv_axons} Ad).
In contrast, the PV+ axon is more irregular. It curves back about 120º shortly after emerging from the soma and branches multiple times (table \ref{tab:pv_vip_axons}, fig. \ref{fig:vip_pv_axons} B). As opposed to the unmyelinated VIP+ axon, the PV+ axon is surrounded by thick myelin sheets (fig. \ref{fig:vip_pv_axons} Ba,b), and does not form synapses nor accumulates vesicles. \\
\captionof{table}{\textbf{Morphological quantification of VIP+ and PV+ interneuron axons.} One VIP+ and one PV+ axon were quantified. Axon length shows the length of axons in the FOV that could be reconstructed and analysed.}
\label{tab:pv_vip_axons}
\begin{tabular}{|l|l|l|l|l|}
	\hline
	Cell type & Axon length (analyzed) & Synaptic output & Myelinated & Branches\\
	\hline
	VIP+ IN & 42.29 $\mu m$ & 2 & No & None\\
	\hline
	PV+ IN & 171.96 $\mu m$ & 0 & Yes & 4\\
	\hline
\end{tabular}
\begin{figure}
	\captionsetup[figure]{indentation=0pt}
	\includegraphics[width=\linewidth]{"vip_pv_axons_Nelo".pdf}
	\caption{\textbf{3D reconstruction of VIP+ and PV+ axons.} \textbf{(A)} Representative image of a VIP+ axon. \textbf{(a and b)} Representative EM images of the axon emerging from the cell body. \textbf{(c and d)} 3D reconstruction and EM images of a synapse found on the VIP+ axon. \textbf{(B)} EM reconstruction of a PV+ axon. \textbf{(a and b)} Representative EM images of the myelinated PV+ axon. Boxes indicate locations of zoom.}
	\label{fig:vip_pv_axons}
\end{figure}
\section{Cilium characterization}
\label{sec:cilium}
Surprisingly, two VIP+ and one PV+ cell have a cilium each, protruding from the soma (fig. \ref{fig:cilium}). The cilia are 8 – 11 µm long (table \ref{tab:cilium}) and are recognizable by the characteristic 9x2+0 arrangement of microtubules typical for non-motile primary cilia (fig. \ref{fig:cilium}A). All reconstructed cilia have a rather regular structure with only minor curves. \\
\captionof{table}{\textbf{Morphological quantification of cilia on VIP+ and PV+ cells.} 2 VIP+ cells and 1 PV+ each present one cilium that could be reconstructed.}
\label{tab:cilium}
\begin{tabular}[pos=t]{|l|l|}
	\hline
	Cell type & Cilium length ($\mu m$)\\
	\hline
	VIP+ & 11.065\\ \cline{2-2}
	& 9.904\\
	\hline
	PV+ & 8.036\\
	\hline
\end{tabular}
\begin{figure}
	\captionsetup[figure]{indentation=0pt}
	\includegraphics[width=\linewidth]{"cilium_Nelo".pdf}
	\caption{\textbf{3D reconstruction and EM micrographs of VIP+ and PV + cilia} 3D EM reconstructions, EM images of the whole cell and high magnification of cilia emerging from VIP+ cells \textbf{(A and B)} and a PV+ cell \textbf{(C)}. Boxes indicate location of zoom. Scale bars = 5~$\mu m$ for low magnification and 1 ~$\mu m$ for high magnification micrographs.}
	\label{fig:cilium}
\end{figure}
