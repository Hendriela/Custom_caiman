\chapter{Methods}
\label{ch:methods}
\section{Animals}
\label{sec:animal}
All experiments were performed according to the guidelines of the Swiss Federal Act on Animal Protection and Swiss Animal Protection Ordinance with licences granted to Professor Carl Petersen.
For this study we employed VIPcre-tdTomato/ChAT-ChR2-EYFP and PVcre-tdTomato/ChAT-ChR2-EYFP transgenic mouse lines. 
The mice were anaesthetised with isoflurane and chemically fixed via transcardial perfusion of 10 ml of isotonic phosphate buffer saline (PBS), followed by 200 ml of 1 \% glutaraldehyde and 2\% paraformaldehyde (PFA) in phosphate buffer (PBS, 0.1 M, pH 7.4).\\ 
Once perfused, animals were left at room temperature (RT) for 2 h. The brains were then removed from the skull and embedded in 5 \% standard agarose (Eurobio) dissolved in PBS. 80~\textmu m thick coronal slices were cut through the region of the primary somatosensory cortex (S1) using a vibratome chamber (Leica VT1200 S).
\section{Fluorescence and light microscopy}
\label{sec:FM}
To visualize tdTomato and EYFP expression in the transgenic mice, an inverted confocal microscope (Zeiss LSM 700 Invert) was used and Z-stacks with 2 µm interplanar distance were acquired at different magnifications (40X, 20X, 10X) to later locate target cells. Emissions of EYFP and tdTomato were recorded at 488 and 546~nm wavelength respectively. VIP/PV expressing cells in layer II/III of S1 with multiple adjacent ChAT+ axons were targeted. Additionally, images were collected using transmitted light to detect slice-specific morphological landmarks such as cell bodies and blood vessels. To maintain high tissue quality and good ultrastructural preservation, the brain slices were kept in PBS throughout the imaging process. To ensure that the slices did not dry out, custom-made chambers consisting of a glass microscope slide, parafilm and a cover glass was used (fig. \ref{fig:chamber}). \\
 \begin{wrapfigure}[11]{r}{0pt}
	\centering
	\includegraphics[width=5cm]{"Imaging_chamber".pdf}
	\caption[Schematic illustration of the imaging chamber]{Schematic illustration of the imaging chamber}
	\label{fig:chamber}	
\end{wrapfigure}
 Afterwards, the slices were brought under a dissecting light microscope (Leica M205 C) containing a camera (Leica MC170 HD). The brain slices were divided along the midline and low magnification images of the entire hemisphere previously imaged under the confocal microscope were collected.\\
 \section{Tissue and block preparation for electron microscopy}
\label{sec:EM} 
\subsection{Tissue embedding}
\label{sec:tissue embedding}
Following the fluorescence and light microscopy, slices were stained and embedded using methods described in Knott et al. (2011). Briefly, the slices were washed in cacodylate buffer (0.1 M, pH 7.4, 3 × 5 min each), post fixed in 2\% osmium tetroxide (Electron Microscopy Sciences, 19110) and 1.5\% potassium ferrocyanide (Sigma 14459-95-1) in cacodylate buffer (0.1 M, pH 7.4, 40 min). They were then stained with 2\% osmium tetroxide in cacodylate buffer (0.1 M, pH 7.4) for 40 min, and then in 1\% uranyl acetate at 4°C overnight. Then the sections were stained in lead aspartate for 2 h, dehydrated in a graded alcohol series for 5 min each, and finally sections were left in Durcupan resin (Electron Microscopy Sciences, 13600) and ethanol solution (1:1) for 30 min. Between each step, the sections were washed in distilled water at RT. After infiltrating overnight, the sections were placed between two glass slides coated in a mold separating agent (Glorex Inspirations, Switzerland; 62407445) and placed in the oven at 65°C for 24 h. 
\subsection{Alignment and block preparation}
\label{sec:block prep}
After Durcupan resin polymerization, the glass slides were separated and the slices stabilized in a thin resin layer. Because sections were impregnated with heavy metals and completely opaque to transmitted light,only their outline could be aligned with the LM images taken before undergoing heavy metal staining (fig. \ref{fig:block_prep}A-C). Overlaying the images acquired before and after heavy metal staining allowed for the location of the regions of interest. These alignments were done in Adobe Photoshop, matching the section edges to identify the region of interest (ROI). The ROI was removed using a razor blade and glued to an aluminium scanning electron microscope (SEM) pin (Micro to Nano place, 10-006002-50) with an electrically conductive epoxy resin (Ted Pella Inc, Redding, CA, USA, 16043). The pin with the sample attached was left overnight at 65°C to polymerize. The following day the block was trimmed using a glass knive mounted in an ultramicrotome (Leica EM UCT). During the trimming, images of the block were imaged frequently to monitor the exact position of the ROI (fig. \ref{fig:block_prep}E).
\begin{figure}[!b]
	\captionsetup[figure]{indentation=0pt}
	\includegraphics[width=\linewidth]{"block_prep".pdf}
	\caption{\textbf{Illustration of the block preparation process.} \textbf{(A and B)} Low magnification (10X) representative images of the tdTomato channel and transmitted light. \textbf{(C and D)} LM images of the same region shown in A and B before and after heavy metal staining. \textbf{(E)} LM images showing different trimming process steps. Blue arrow indicates location of reconstructed cell in the ROI. Purple dashed lines show capillaries used as landmarks. Scale bars = 500~\textmu m.}
	\label{fig:block_prep}
\end{figure} 
\section{Serial Block-Face EM imaging}
\label{sec:3View}
Before loading the sample into the SEM microscope (Merlin, Zeiss NTS) fitted with the 3View cutting system (Gatan, Inc., Pleasanton, CA, USA), a 40~nm gold coating was applied to the block surface. This is necessary to improve the conduction of the surface and limit imaging artifacts caused by the build up of electrical charge. The first cuts were conducted with an open microscope chamber to remove the gold coating. Then the chamber was closed and pumped to a high vacuum. Next, a low resolution scanning of the block face was acquired to reveal the position of blood vessels seen in the LM images (fig. \ref{fig:block_overviews}B, D). To determine the exact ROI, a final overlay of the EM block face was made with the LM images. At this point we proceeded to cut away sections from the surface of the block, constantly monitoring the surface in relation to the height of the ROI. For imaging following settings were used: EHT = 1.7~kV; current beam = 300~pA; knife cutting speed = 0.3~mm/sec; backscatter detector = on; image size = 6000 x 6000 pixels, resolution = 6.5~nm x and y; slice thickness = 50~nm. Several tiles per slice were aquired for a larger field of view.
\begin{figure}[!h]
	\captionsetup[figure]{indentation=0pt}
	\includegraphics[width=\linewidth]{"block_overviews".pdf}
	\caption{\textbf{Low resolution block face images.} \textbf{(A)} Transmitted light image showing the pattern of blood vessels on the slice. \textbf{(B)} Block face overview at low resolution showing blood vessels and cell bodies. \textbf{(C)} Low resolution EM image of the block face at a lower z-depth. Dashed lines indicate landmark capillaries, arrows show target cell. Scale bar = 500~\textmu m.}
	\label{fig:block_overviews}
\end{figure}
\section{EM image processing, segmentation and 3D model generation}
\label{sec:fiji}
The raw stacks of images collected from the microscope were first aligned in Fiji with the TrackEM 2 plugin. The alignment function exported the stack in a xml file which was opened in TrackEM 2 to perform the segmentation process. The same plugin was used to trace and segment the features of interest, and create the 3D models. Once the tracing was completed, 3D models were exported as obj files and further processed and analyzed in Blender (www.blender.org). 
\section{Blender and 3D model processing}
\label{sec:blender}
Each segmented object was drawn as an area list in TrackEM 2 and imported into the Blender workspace running the Neuromorph extension (Jorstad et al. 2018). This implementation enable various measurement to be made, such as of surfaces area,, lengths and volumes. Here it was used to measure synapse area, cell body volumes and various lengths.  
\section{Statistical analysis}
\label{sec:stat}
Neuromorph measurements were processed and analysed with Microsoft Excel 2019. Statistical analysis and plotting was performed with Prism7 (GraphPad Software, San Diego, USA). Synaptic input was analyzed with the excitatory input ratio which was calculated with the following formula: $\frac{N_{ex} / l_{d}}{N_{inh} / l_{d}}$, with $N_{ex}$ and $N_{inh}$ being the numbers of excitatory and inhibitory synapses, which are both normalized against the total length of reconstructed dendrites $l_{d}$. Results are reported as mean +/- SEM. Statistical tests are described in the respective figure legends.  Statistically  significant P-values are displayed in graphs as: * P<0.05, ** P<0.01, *** P<0.001. 

