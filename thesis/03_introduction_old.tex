\chapter{Introduction}
\label{ch:intro}
\acresetall
\section{Barrel cortex and GABAergic inhibitory interneurons}
\label{sec:GABA neurons}
\subsection{Barrel cortex structure and function}
\label{subsec:bcx1}
The neocortex is the most complex part of the brain responsible for the execution of higher-order brain functions, such as cognition, sensory perception and sophisticated motor control, and deciphering how information is coded and processed in the is one of the greatest challenges in neuroscience (\citep{Lodato2016}). The neocortex is organized in six layers, each with distinct cell types and connectivity. In the rodent primary somatosensory cortex (S1), also called barrel cortex, the six layers of neuronal tissue are organized in juxtaposed and interconnected columns. A cortical column is considered the basic module of cortical information processing present in all cortical areas and formed through a characteristic microcircuit composed of a few thousand neurons (REFERENCE). \\
The vibrissae of rodents are moved actively to touch – requiring intricate sensorimotor interactions. On the microscopic level, the cortical modules of the rodent S1 process somatosensory inputs at highest magnification and discreteness – each whisker is represented by its own so-called barrel column \citep{Feldmeyer2013}. In the barrel cortex, these columns are defined by a readily discernible patch-like structure in layer 4. Information processing in the neocortex occurs along vertical and horizontal axes, thereby linking individual barrel columns via axons running through the different cortical layers of the barrel cortex (linking columns by axons that go through different layers? but barrels are in layer 4...). Long-range signalling occurs within the neocortical layers but also through axons projecting through the white matter to other neocortical areas and subcortical brain regions. Because of the ease of identification of barrel-related columns, the rodent barrel cortex has become a prototypical system to study the interactions between different neuronal connections within a sensory cortical area and between this area and other cortical as well subcortical regions \citep{Feldmeyer2012}.
\subsection{Barrel cortex cell composition} 
The rodent barrel cortex is constituted of excitatory and inhibitory neurons. In each barrel column  80 \% of the neurons are excitatory while 20 \% is inhibitory (REFERENCE). The excitatory neurons are homogeneous pyramidal cells that use glutamate as a neurotransmitter. The inhibitory components, on the contrary, are extremely diverse regarding morphological, organizational, electrophysiological and genetic aspects. Cortical inhibition is mainly mediated by the GABAergic interneurons. Different types of GABAergic interneurons strongly govern the activity of cortical circuits towards meaningful behavior by feed-forward and feedback inhibition as well as disinhibition \citep{Staiger2015}. A first classification is made according to non-overlapping markers such as parvalbumin (PV, a Ca\textsuperscript{2+}-binding protein, \textasciitilde40\%), somatostatin (SST, \textasciitilde30\%) and the ionotropic serotonin receptor 5HT3a (5HT3aR, \textasciitilde30\%). Each group has a different embryological origin and contains further subdivisions \citep{Staiger2015,Tremblay2017,Rudy2011}. \\
These inhibitory interneurons are thought to contribute to a variety of cortical functions such as gamma oscillatory activity (Mushiake 2015), learning (Li et al. 2015), plasticity (Fu et al. 2015) etc. Malfunctioning of inhibitory neurons are also implicated in a variety of pathological conditions like epilepsy (Cobos et al. 2005), schizophrenia (Rogasch, Daskalakis, and Fitzgerald 2014) and bipolar disorder (Levinson et al. 2007). Therefore, gaining an understanding of cortical connections in the neocortex through the characterization of the different cell type populations and how they are coded and modulated into circuits is a major focus of research.
Identifying the cortical connections underlying the inhibitory neurons of the barrel cortex continues to be a major field of research. Apart from their modulatory effect on pyramidal neuron activity, direct interactions between inhibitory interneurons have been implicated in sensory processing as well. 
\begin{figure}[!h]
	\captionsetup[figure]{indentation=0pt}
	\includegraphics[width=\linewidth]{"barrels".png}
	\caption[Schematic illustration of barrel cortex organization.]{\textbf{Schematic illustration of barrel cortex organization.}}	
	\label{fig:barrel_cortex}
\end{figure}
\section{VIP-expressing interneurons }
\label{sec:VIP}
\begin{figure}[!h]
	\captionsetup[figure]{indentation=0pt}
	\includegraphics[height=.8\textheight]{"GABA_classification".pdf}
	\caption[Scheme showing the different types of GABAergic interneurons.]{\textbf{Scheme showing the different types of GABAergic interneurons.} \textbf{(b-d)} Overview of the barrel cortex in PVcre section (green), SSTcre section (yellow) and VIPcre section (red). \textbf{(e-g)} Reconstruction of a PV-expressing, fast-spiking basket cell, a somatostatin-expressing, adapting Martinotti cell and a VIP-expressing, irregular-spiking bitufted cell. \textbf{(h-j)} Raffiguration of action potential firing pattern. (Modified from Staiger et al., 2015)}	
	\label{fig:gaba_classes}
\end{figure}
 
\subsection{VIP+ interneuron function and distribution across barrel cortex}
\label{subsec:vip distribution}
Neurons expressing VIP belong to the ionotropic serotonin receptor (5HT3aR)-positive cells that produce $ \gamma $-aminobutyric acid (GABA). 
Cortical VIP neurons are key elements in neurovascular coupling (Cauli et al. 2004) and in the regulation of neuronal energy metabolism (Magistretti et al. 1998). In terms of cortical circuitry, it has been repeatedly reported that VIP neurons preferentially target several other types of inhibitory interneurons (Staiger et al. 2004b; David et al. 2007; Pfeffer et al. 2013), thus mediating disinhibition by releasing excitatory principal neurons from in- hibition. The disinhibitory circuit motif involving the VIP-to- SOM connectivity has now been functionally characterized in the visual (Fu et al. 2014;  Zhang et al. 2014), the auditory (Pi et al. 2013), and the primary somatosensory (barrel) cortex (Lee et al. 2013).
The VIP neurons in the barrel cortex are not homogenously distributed across layers with densities ranging from 44.6 ± 40.5 cells/mm3 cortex in layer I to 1366.6 ± 285.8 cells/mm3 in layer II/III. This compartment presents the highest VIP-positive cell bodies consisting of the 58.7\% of the total VIP cells in the barrel cortex (Prönneke et al., 2015). 
Bitufted cells, which are only found in layers II–V, have an elongated soma that generally lies perpendicular to the pia, with most of the dendritic processes emerging from the top and bottom of the soma.
\subsection{Morphological and electrophysiological characteristics of VIP+ cells }
\label{subsec:vip2}
The VIP-positive interneurons are found in every layer of the barrel cortex, thus their morphological organization depends on the layer they are positioned. As concerning the cell body, it is generally elliptic with a larger vertical then horizontal diameter. As for their dendritic organization, there are two main families of VIP+ cells: the bipolar and bitufted. VIP+ cells are highly diverse, thus VIP+ with multipolar or
\section{The PV-expressing interneurons}
\label{sec:pv}

\section{Cortical acetyl cholinergic input}
\label{sec:chat}
Acetylcholine from the basal forebrain and from intracortical inhibitory interneurons exert strong influence on the cortical activity and may interact with each other.
\section{3D electron microscopy}
\label{sec:EM microscopy}