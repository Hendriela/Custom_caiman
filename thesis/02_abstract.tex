\chapter*{Abstract}
\label{ch:abstract}
\addcontentsline{toc}{chapter}{Abstract}
The neurotransmitter acetylcholine has wide-ranging effects on brain functions, by modulating and shaping neuronal activity, synaptic transmission and plasticity. The main source of cholinergic axons in the cortex is the basal forebrain, whose neurons project widely across the brain. However, despite physiological evidence that suggests a direct effect of acetylcholine on the interneurons that inhibit pyramidal neurons, there is no structural evidence of these specific synaptic contacts. To address this question and characterize the nature of cholinergic contacts onto interneurons, we used a correlative light and electron microscopy technique to target two types of inhibitory cell types: vasoactive intestinal peptide (VIP) and parvalbumin (PV) expressing interneurons, and investigated their interactions with cholinergic axons. We reconstructed and characterized the VIP- and PV-expressing cells, analysing their synaptic inputs and possible connections with cholinergic axons. Results show that dendritic spines are absent from PV+ neurons with only few spines on the VIP+ cell, although both had similar numbers of excitatory inputs on the dendritic shaft. Regarding the cholinergic connectivity with these cell types, axons appear to be closely apposed to the VIP+ and PV+ interneurons, particularly at their somata. However, there is limited evidence of typical synaptic connections with clear features that would allow us to classify these sites as synapses. Nevertheless, the abundance of vesicles and contacts with the membranes could represent regions of possible communication. This finding suggests that the connectivity between the cholinergic neurons and these interneurons in the cerebral cortex may not appear as typical synaptic connections, raising questions about how the connectivity of neuromodulatory axons should be characterized.
 
