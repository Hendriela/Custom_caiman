\chapter{Introduction}
\label{ch:intro}
\acresetall
\section{The barrel cortex and GABAergic inhibitory interneurons}
\label{sec:GABA neurons}
\subsection{Barrel cortex structure and function}
\label{subsec:bcx1}
The neocortex is a six-layered structure in the brain responsible for the execution of higher-order brain functions, such as cognition, sensory perception and sophisticated motor control. Deciphering how information is coded and processed in the neocortex is one of the greatest challenges in neuroscience \citep{Lui2011,Lodato2016}.\\
The rodent primary somatosensory cortex (S1) or barrel cortex is one of the main model systems for cortical circuitry. Somatosensory information from the vibrissae reach the barrels, dense patch-like structures in layer IV. These barrels have a somatotopic organization, where each whisker is represented by its own barrel column \citep{Feldmeyer2013a, Petersen2013a}. Its accessibility, distinct anatomy and comparably easy manipulation make the barrel cortex a well studied system for the complex connectivity and functionality of neocortical networks. When investigating basic microcircuits, working in the barrel cortex offers the benefit of being able to have recourse to decades of research on its constitution and circuitry, making it possible to isolate fine aspects of a complex neuronal system. Afterwards, findings of basic research on the barrel cortex can be generalized to other, less understood cortical regions.
\subsection{Barrel cortex cell composition} 
The rodent barrel cortex is constituted of excitatory and inhibitory neurons. In each barrel column  80\% of the neurons are excitatory while 20\% is inhibitory \citep{Markram2004}. The excitatory neurons are homogeneous principal cells (PCs) that use glutamate as a neurotransmitter. The inhibitory components, on the contrary, are extremely diverse regarding morphological, organizational, electrophysiological and genetic aspects. Cortical inhibition is mainly mediated by the GABAergic interneurons. Different types of GABAergic interneurons strongly govern the activity of cortical circuits towards meaningful behavior by feed-forward and feedback inhibition as well as disinhibition (fig. \ref{fig:gaba_classes}, \cite{Staiger2015}). A first classification is made according to non-overlapping markers such as parvalbumin (PV, a Ca\textsuperscript{2+}-binding protein, \textasciitilde40\%), somatostatin (SST, \textasciitilde30\%) and the ionotropic serotonin receptor 5HT3a (5HT3aR, \textasciitilde30\%). Each group has a different embryological origin and contains further subdivisions \citep{Staiger2015,Tremblay2017,Rudy2011}. \\
These inhibitory interneurons are thought to contribute to a variety of cortical functions such as gamma oscillatory activity, learning and plasticity \citep{Kuki2015,Fu2015,Li2015}. Malfunctioning of inhibitory neurons are also implicated in a variety of pathological conditions like epilepsy \citep{Cobos2005}, schizophrenia \citep{Rogasch2014} and bipolar disorder \citep{Levinson2007}. Therefore, gaining an understanding of cortical connections in the neocortex through the characterization of the different cell type populations and how they are coded and modulated into circuits remains a major focus of research.
\begin{figure}[!h]
	\captionsetup[figure]{indentation=0pt}
	\includegraphics[width=\linewidth]{"barrels".png}
	\caption[Schematic illustration of barrel cortex organization.]{\textbf{Organization of the barrel cortex.} The barrels are dense patches in layer IV of the rodent primary somatosensory cortex. Each barrel receives input from one whisker in a somatotopic organization. Aside from the dense connectivity in each barrel, cross-talk between barrels as well as modulation from other brain areas occurs. From \cite{Feldmeyer2013a}.}	
	\label{fig:barrel_cortex}
\end{figure}
\section{The VIP-expressing interneurons }
\label{sec:vip distribution}
\begin{figure}[!h]
	\captionsetup[figure]{indentation=0pt}
	\includegraphics[height=.8\textheight]{"GABA_classification".pdf}
	\caption[Scheme showing the different types of GABAergic interneurons.]{\textbf{Scheme showing the different types of GABAergic interneurons.} \textbf{(b-d)} Overview of the barrel cortex in tissues expressing markers for PV (green), SST (yellow) and VIP (red). \textbf{(e-g)} Reconstructions of a PV-expressing, fast-spiking basket cell, a somatostatin-expressing, adapting Martinotti cell and a VIP-expressing, irregular-spiking bitufted cell. \textbf{(h-j)} Representation of action potential firing patterns of aforementioned cell types. From \cite{Staiger2015}.}	
	\label{fig:gaba_classes}
\end{figure}
 
Neurons expressing VIP belong to the ionotropic serotonin receptor (5HT3aR)-positive cells that produce $ \gamma $-aminobutyric acid (GABA). Cortical VIP neurons are key elements in neurovascular coupling \citep{Cauli2004} and in the regulation of neuronal energy metabolism \citep{Magistretti1999}. In terms of cortical circuitry, it has been repeatedly reported that VIP neurons preferentially target several other types of inhibitory interneurons \citep{Staiger2004,David2007,Pfeffer2013}, thus mediating disinhibition by releasing excitatory principal neurons from inhibition (fig. \ref{fig:disinhib_circuit}, \cite{Wang2018}). The resulting disinhibitory circuit motif involving the VIP-to-SST connectivity has now been functionally characterized in the visual \citep{Fu2014,Zhang2014}, the auditory \citep{Pi2013}, and the S1 barrel cortex \citep{Lee2013, Wang2018}. \\
The VIP neurons in the barrel cortex are not homogenously distributed across layers with densities ranging from 44.6 ± 40.5 cells/mm\textsuperscript{3} cortex in layer I to 1366.6 ± 285.8 cells/mm\textsuperscript{3} in layer II/III. This compartment presents the highest number of VIP-positive cells, accommodating 58.7\% of the total VIP cell number in the barrel cortex \citep{Proenneke2015}. \\ 
The morphological organization of VIP+ interneurons depends on the layer they are located in. The soma is generally elliptic with a larger vertical than horizontal diameter. VIP+ cells can be categorized in two groups by their dendritic structure: bipolar and bitufted cells.The former have two big dendrites with few branches that emerge on opposite sides of the cell body. The latter have more irregular features and usually more than 2 dendrites (\cite{Feldmeyer2013a}, see fig. \ref{fig:vip_cell_body})
\section{The PV-expressing interneurons}
\label{sec:pv}
The PV+ interneurons express the calcium-binding protein parvalbumin (PV),  and constitute with 40\% the largest group of GABAergic interneurons \citep{Celio1986}. These cells have a high density in layer IV and present two main subclasses: the basket and axon-axonic (or chandelier) cells (fig. \ref{fig:gaba_classes}, \cite{Markram2004}). PV+ interneurons have multiple dendrites which are highly ramified and distributed across layers which allows these interneurons to receive input from different afferent pathways, such as feedforward and feedback pathways \citep{Hu2014}. Electrophysiologically, these cells are characterized by a fast spiking activity due to the expression of AMPA-like glutamate receptors with high Ca\textsuperscript{2+} permeability and fast gating \citep{Kawaguchi1987,Geiger1995}. The PV+ cells modulate the perisomatic activity of the PCs by controlling their spiking output \citep{Tremblay2017,Wang2017}. As mentioned in chapter \ref{sec:vip distribution}, the cortical interneurons form an disinhibitory circuit where the PV+ cells play an important role innervating the PC's cell body.

\section{Acetylcholine in the cortex}
\label{sec:chat}

 \begin{wrapfigure}[14]{r}{0pt}
	\centering
	\includegraphics[width=8cm]{"ACh".png}
	\caption[ACh distribution in the cortex]{\textbf{Acetylcholine distribution in the cortex}. Acetylcholine is produced by a few brainstem and basal forebrain nuclei. From there, projecting axons distribute ACh throughout the CNS. From \cite{Thiele2013}.}
	\label{fig:ach}	
\end{wrapfigure}

Recently, acetylcholine (ACh), the main neurotransmitter at neuromuscular junctions and in the autonomous nervous system, has been found to modulate and shape the neuronal activity, the synaptic transmission and the synaptic plasticity of the central nervous system (CNS; \cite{Alitto2013,Fu2014,Picciotto2012a}). ACh is mainly produced by cholinergic neurons of the basal forebrain (BF) that project to different areas of the cortex including the barrel cortex (fig. \ref{fig:ach}; \cite{Thiele2013}). These cholinergic projections enhance or inhibit a circuit where the VIP+ cells inhibit the SST+ cells, which in turn disinhibits the excitatory PCs \citep{Pfeffer2013}. ACh acts on two different receptors: muscarinic (mAChR) and nicotinic (nAChR) receptors. When ACh acts on the mAChR, it induces activity of PCs and PV+ interneurons, whereas binding to nACh activates VIP+ cells \citep{Alitto2013}. This divergent effect of ACh is correlated to the resulting cortical neuronal state: synchronization (mACh activation, quiet wakefulness) and desynchronization (nACh activation, active behavior such as whiskering; \cite{Crochet2006,Eggermann2014}). \\ 
Recently, a study of the disinhibitory circuit in the S1 showed that only a specific subtype of SST+ interneurons is inhibited during whisking behavior. This SST+ suppression was mediated by VIP+ cells \citep{Munoz2017}. A critical finding for our project is the ACh-induced enhancement of VIP+ activity in the primary visual cortex (V1, \cite{Fu2014,Pinto2013}). By using in-vivo calcium imaging, it was discovered that during locomotion VIP+ interneurons were active, independently of visual stimuli \citep{Fu2014}. At the same time, SST+ cells were inactive. Additionally, optogenetic stimulation of the BF cholinergic neurons enhanced visual perception and cortical activity in V1 \citep{Pinto2013}. These findings support the circuit proposed by \cite{Pfeffer2013} where VIP+ interneurons enhance the nearby PCs by inhibiting SST+ neurons (fig. \ref{fig:disinhib_circuit}).\\

\begin{figure}
	\floatbox[{\capbeside\thisfloatsetup{capbesideposition={right,top},capbesidewidth=6cm}}]{figure}[\FBwidth]
	{\caption{\textbf{Proposed cortical disinhibitory microcircuit.} In this model, SST+ neurons inhibit distal dendrites and PV+ cells target the soma of excitatory pyramidal cells (grey), suppressing excitatory output of the microcircuit. VIP+ interneurons target SST+ cells. Thus, VIP+ activity inhibits SST+ neurons, lifting their suppression on principal cells and increasing the excitation of the circuit via disinhibition. The precise connectivity, especially of PV+ interneurons, is still under debate. Other studies have shown that ACh released from ChAT+ axons, originating in basal forebrain nuclei, can influence the activity of VIP+ cells, thus possibly modulating the entire circuit \citep{Alitto2013}. However, the nature of communication between ChAT+ axons and VIP+ neurons is still unclear. Adapted from \cite{Wang2018}.}
	\label{fig:disinhib_circuit}}
	{\includegraphics[width=9cm]{"disinhibitory_ciruit".pdf}}	
\end{figure}

Although the effect of ACh in the cortex has been shown, no study has reported cholinergic synapses in the barrel cortex. Instead, ACh is thought to be released via volume transmission, release ACh untargeted into the extracellular matrix (ECM). However, ACh has been shown to act faster on target cells than would be expected for volume transmission \citep{Fu2014}. The hypothesis of this project is that there are cholinergic synapses on the L2/3 VIP+ neurons in the barrel cortex. To demonstrate this, we are going to use correlated light and electron microscopy (CLEM).
\section{3D electron microscopy}
\label{sec:EM microscopy}
Since the 1930s, electron microscopy (EM) has played an important role in describing the ultrastructure of biological samples \citep{Knott2013}. Therefore, various techniques have been developed to overcome some of its limitations such as structure-obscuring immunolabeling, narrow fields of view, volume imaging and fixation-induced tissue shrinkage. In this project, we avoid antibody-labelling of structures by using transgenic mice that express fluorescent proteins in the cells of interest. Moreover, we are using a correlative light and electron microscopy (CLEM) method developed by \cite{Maclachlan2018a}. Briefly, characteristic tissue landmarks such as cell bodies or capillaries are identified via light microscopy, which are then used to locate the same region under the electron microscope. In order to create large-scale reconstructions of neuronal structures and to obtain 3D models of the reconstructed cells, we use a serial block face scanning EM (SBEM) equipped with the 3View system (see ch. \ref{sec:3View}). This is a automated volumetric EM technique where thin sections are constantly removed from the resin block surface, yielding a nearly complete reconstruction of the complete sample block. Additionally, several tiles of the same surface can be imaged and later aligned to increase the originally narrow field of view. These techniques provide a high-resultion volumetric dataset the sample free from immunostaining.
\section{Aim of the project}
\label{sec:aim}
Acetylcholine has been shown to affect VIP+ interneurons with a specificity and temporal resolution presumably too high for pure volume transmission, although synapses of ChAT+ axons onto cortical GABAergic interneurons have not been shown yet.
We hypothesize that the cholinergic axons projecting from the BF are making synapse-like connections onto the GABAergic interneurons of the barrel cortex. We will focus on L2/3 VIP+ and PV+ cells. To assess our hypothesis, we will use two transgenic mouse lines where EYFP and tdTomato mark the ChAT+ axons and VIP+ or PV+ cells respectively. These structures are imaged simultaneously, allowing for interaction studies. To understand these connections, we will use correlated light and electron microscopy (CLEM) where regions of interest imaged in fluorescence microscope are assessed at a ultrastructural level via electron microscopy and later reconstructed to 3D models. Besides characterizing these reconstructed neuronal structures, this technique will allow us to investigate the connectivity of cholinergic axons onto GABAergic interneurons at nanoscale resolution.