\chapter{Discussion}
\label{ch:discussion}
\section{Small sample size limits significance}
As stated previously, the data that could be gathered during this project is originating from a very limited sample size. In total, 4 VIP+ neurons from 2 mice, 1 PV+ cell as well as 7 ChAT+ axons from 2 mice were reconstructed. Whereas the quality of the dataset itself is not affected, this small sample size makes it difficult to generalize our findings to the whole population of the investigated cell types. One has to bear this in mind when drawing conclusions from the results presented here.
\section{VIP+ and PV+ soma and nuclei are irregularly shaped}
\label{sec:vip_pv_nuclei}
In this project, we imaged and reconstructed 4 VIP+ and 1 PV+ interneuron in layer II/III of the rodent barrel cortex. Through reconstructing cells imaged with a scanning electron microscope to digital 3D models, we observe that VIP+ cells appeared to be more elliptic (not measured directly) and smaller in size compared to PV+ cells. More specifically, the PV+ soma has a volume three times higher than the VIP+ neurons. In both cases the nuclear membrane foldings were observed (fig. \ref{fig:vip_cell_body}, \ref{fig:pv_cell_body}). Neuronal nuclei have been shown to change their morphology according to the activity of N-methyl-D-aspartate receptors (NMDARs, \cite{Wittmann2009}), as membrane foldings are indicators of an elevated interaction between endoplasmic reticulum (ER) and the nucleus. NMDARs are well known to drive long term plasticity (LTP) in neurons, where activation of NMDARs leads to increased calcium influx at the postsynapse, which in turn activates signalling cascades inducing new protein synthesis \citep{Malenka1991}. Therefore, our results suggest that these cell types could have a high synaptic activity and plasticity, indicating an important role in the modulation of the activity of the microcircuit in which they are situated. It would be interesting to understand if more finely grained structural details like ER arrangements around the nucleus are also present, and whether these features are also found in other interneurons with similar disinhibitory roles, such as SST+ cells. Unfortunately, this was outside the scope of this project.  
\section{VIP+ and PV+ dendrites receive excitatory and inhibitory input}
We analyzed around 300 (proximal and distal) and 327 $\mu m$ of VIP+ and PV+ dendrites respectively (fig. \ref{fig:vip_dendrites}, \ref{fig:pv_dendrites}, table \ref{tab:vip_dendrites}, \ref{tab:pv_dendrites}). Synapse reconstructions revealed a higher excitatory than inhibitory input for both cell types. These synapses are generally found on the dendritic shaft, and few spines were reconstructed (total n = 5, 4 VIP+, 1 PV+), indicating that these cells are aspiny neurons (fig. \ref{fig:vip_dendrites}, \ref{fig:pv_dendrites}). Furthermore, analysis of synapses on the proximal and distal VIP+ dendrites revealed substantial disparities in their synaptic input in terms of the proportion of excitation to inhibition.The inhibitory inputs, although slightly smaller than the excitatory inputs, are more abundant on distal compared to proximal dendrites (fig. \ref{fig:vip_dendrites}E, table \ref{tab:vip_dendrites}). In fact, total excitatory input (as measured as the ratio between excitatory and inhibitory synapse numbers) is 25 times greater than inhibitory input for proximal and 5 times greater for distal dendrites, a 5-fold difference between proximal and distal dendrites (table \ref{tab:vip_dendrites}). While this analysis is based on synapse numbers, the conclusion holds when calculating the ratio of excitatory versus inhibitory synapse sizes, although the result of a 4-fold difference (table \ref{tab:vip_dendrites}) is somewhat less pronounced. \\
This variation in the inhibitory input suggests a high targeting specificity of axons from different origins. Proximal dendrites are mainly responsible for the generation of action potentials (APs), as they are closer to the axon hillock. The hillock is considered the AP-generating region of the soma, thus synaptic currents on proximal dendrites have a larger influence on the AP generation by being more closely located to the axon hillock \citep{Palay1968,Wollner1986, Haeusser2000}. Synapses on the distal dendrites however have more modulatory function, indirectly related to APs. They instead influence the activity of the postsynaptic neuron by silencing or disinhibiting regions of the dendritic membrane. These results suggest that even inhibitory interneurons - being themselves modulatory for principal neurons – receive dense modulating inhibitory input, supporting the theory of local disinhibitory circuit motifs as an essential aspect of cortical networks \citep{Pfeffer2013,Pfeffer2014}.
\section{Comparing synapse sizes of proximal and distal VIP+ and PV+ dendrites}
Classically a synaptic connection is defined as a specific structure with a pre-element containing neurotransmitter vesicles, and a post-element receiving the neurotransmitter. Glutamatergic and GABAergic synapses have been well characterized and are easily recognizable on  EM images. However, the modulatory connections are not as well defined. In our case, we traced glutamatergic and GABAergic synapses on proximal/distal VIP+ and PV+ dendrites (ch. \ref{sec:VIP dendrites}, fig. \ref{fig:vip_dendrites}, \ref{fig:pv_dendrites}). \\ 
We found that proximal GABAergic and glutamatergic synapses on VIP+ cells have comparable surface area (respectively 0.27 $\pm$ 0.04 and 0.25 $\pm$ 0.14~$\mu m^2$, fig. \ref{fig:vip_dendrites}D). Similar observations were made also on the distal dendrites where the GABAergic and the glutamatergic synapses have a surface area of 0.18 $ \pm $ 0.13 and 0.19 $ \pm $ 0.10 respectively. However, when proximal and distal synapse size were compared between each other, we saw a reduction in size of the synaptic surface, with a significant difference between proximal and distal glutamatergic synapses. \\ 
Similarly, PV+ dendrites form both GABAergic and glutamatergic synapses (table \ref{tab:pv_dendrites}). In contrast to VIP+ neurons, the surface areas of GABAergic and glutamatergic synapses on PV+ cells are significantly different (ch. \ref{sec:PV dendrites}, fig. \ref{fig:pv_dendrites}), suggesting that there might a higher release of glutamate. However, substantiating this hypothesis will require further analysis and recordings with higher resolution to more accurately identify and reconstruct synapse types.\\
\section{Cholinergic axons form synapse-resembling connections}
\label{sec:ach_axons}
The barrel cortex is supplied with acetylcholine mainly by cholinergic axons projecting from the nucleus basalis (Ch4 or basal forebrain, fig. \ref{fig:ach}). However, local cortical interneurons have been shown to express ChAT, with 98\% of this cortical cholinergic population coexpressing VIP (fig. \ref{fig:chat_population}, \cite{Gonchar2008}). This phenomenon would partially explain why we observe a substantial amount of VIP+/tdTomato signal in the ChAT/EYFP channel (fig. \ref{fig:chat_axons}A, \ref{fig:vip_chat}A). This is enhanced by overlapping excitation spectra of EYFP and tdTomato. For our study, we focused on axons that emitted only in the EYFP channel, which should eliminate 98\% of local cholinergic interneurons. We can thus be fairly confident that the reconstructed ChAT+ axons are originating from the basal forebrain and are not ChAT-expressing VIP+ cells.
\begin{figure}
	\captionsetup[figure]{indentation=0pt}
	\includegraphics[width=\linewidth]{"chat_population".png}
	\caption{\textbf{Protein expression of cortical interneuron populations.} Relevant for this study is the expression of ChAT in GABAergic interneurons. As can be seen in this graphic, ChAT is only expressed by GABAergic interneurons exclusively in co-expression with VIP. From \cite{Gonchar2008}.}
	\label{fig:chat_population}
\end{figure}
The analyzed ChAT+ axons showed previously reported features. They are thin, unmyelinated, highly ramified and form vesicle-accumulating varicosities \citep{Descarries2000}. The ChAT+ axons were analyzed with the focus to understand how they are acting on other cortical elements. There is an increasing interest in understanding these mechanisms because recent studies have shown that acetyl choline is involved in the GABAergic interneurons circuitry modulation \citep{Fu2014}. Specifically, ACh is hypothesized to enhance the VIP+ activity which inhibits SST+ interneurons. The SST+ cells innervate the distal dendrites of pyramidal cells (PCs), therefore their inhibition results in a disinhibition of excitatory PCs, causing a net increase in excitatory network activity (fig. \ref{fig:disinhib_circuit}). Moreover, functional studies have shown a rapid transmission mechanism suggesting that the cholinergic axons might make direct contacts with the VIPs. Studies have reported evidence for both volume transmission and synaptic contacts of ChAT+ axons \citep{Sarter2009}. Our findings confirm that ChAT+ axons are very closely associated with interneurons, but we were not able to clearly identify true synapses. However, we observed appositions of the cholinergic axons onto dendritic shafts and spines (fig. \ref{fig:chat_axons}C). Although these structures do not perfectly resemble classic synapses with a synaptic cleft, pre- and post-synaptic densities and docked vesicles on the pre-synaptic membrane, we still observe these features albeit less accentuated. In fact, these synapse-like contacts present a smaller surface area (\textasciitilde 0.085 $\mu m^2$) and are less frequent (0.16 contacts/\textmu m (table \ref{tab:chat axons}) than GABAergic and glutamatergic synapses (tables \ref{tab:vip_dendrites}, \ref{tab:pv_dendrites}). These synaptic-like structures were found every \textasciitilde 6 $\mu m$ along the analyzed axons.

\section{ChAT+ axons contact VIP+, but not PV+ interneurons}
\label{sec:chat_gaba}
A question that is still discussed vigorously is the nature of ACh communication with local interneurons \citep{Sarter2009}. Whereas is has been reported that ChAT+ axons are in principal able to form synaptic contacts \citep{Turrini2001}, synapses between ChAT+ axons and GABAergic interneurons have not yet been detected, although communication between them has been shown repeatedly \citep{Wanaverbecq2007,Pfeffer2013,Fu2014,Letzkus2011}. It is thus hypothesized by some that ACh is released via volume transmission rather than through true synapses \citep{Descarries1997}. However, in a recent study conducted by Fu et al. \citeyear{Fu2014} in the visual cortex, ACh was found to activate the VIP+ cells directly through nicotinic ACh receptors (nAChR). Moreover, some studies reported cholinergic activity onto PV+ interneurons \citep{Letzkus2011}. Here, we attempt to clarify the interaction between cholinergic axons and cortical GABAergic interneurons by co-labelling ChAT and VIP or PV. Correlative light and electron microscopy (CLEM) makes it possible to locate previously imaged cells in EM images without the use of fiducial landmarks or EM staining techniques, both of which considerably impact tissue quality and resolution. CLEM is particularly useful to reconstruct small nanoscale structures such as spines and synapses. This allows us to reconstruct ChAT axons as well as find, identify and analyze contacts between it and VIP+ or PV+ cells.\\
Along the \textasciitilde 333 $\mu m$ of reconstructed ChAT+ axons we observed 4 contacts between cholinergic axons and VIP+ cells. We found membrane approximations at the level of the cell body, proximal and distal dendrites (fig. \ref{fig:vip_chat}). However, these contacts are not clearly synaptic-like. Postsynaptic densities are missing, although varicosities accumulate neurotransmitter in large numbers of vesicles (fig. \ref{fig:vip_chat}E,F). While here ACh would be released via volume transmission, due to the proximity of the VIP+ cell it is possible that it would receive most of the transmitter, providing a certain spatial specificity to the volume transmission.\\
In contrast to the contacts observed between ChAT+ axons and VIP+ cells, no such connection was found on the PV+ neuron. Although the ChAT+ axon was found to be directly adjacent to the PV+ soma, no vesicle accumulation was detected in this area. Thus, a functionality of this tangent is unlikely. This finding would corroborate the proposed disinhibitory circuit (fig. \ref{fig:disinhib_circuit}, \cite{Pfeffer2013}) as well as previous studies showing an effect of ACh on VIP+ neurons, but neither PV+ nor SST+ cells \citep{Alitto2013}.

\section{ECS shrinkage might influence neuronal contact morphology}
\label{sec:ECS}
VIP+ cells have been shown to directly contact ChAT+ axons at vesicle-accumulating varicosities (fig. \ref{fig:vip_chat}). In contrast, we found only one cholinergic axon contacting the PV+ cell body. This region showed no clear vesicle accumulation (fig. \ref{fig:pv_chat}C-F). These tangent membranes without any synaptic-like stuctures could be due to fixation artifacts. In fact, it has been well established that the chemical fixation performed for EM imaging causes tissue shrinkage \citep{Boyde1980}. Therefore, extracellular space (ECS) reduction might artificially bring the two elements closer together, without a natural proximity.\\
Previous analyses of specific types of connections in the CNS have relied on immunocytochemistry where antibodies label the structures of interest \citep{Bopp2017}. Although this method allows the different types of neurons to be easily identified in the tissue, it does not provide optimal preservation of the ultrastructure, as the tissue needs to be treated to provide access for antibodies and stains. This compromises the ultrastructure and leads to nanostructures like synapses being obscured. By using a correlative approach, no specific marking of the axons or dendrites of interest was needed to identify them in the EM, enabling the choice of fixation and EM staining techiques for optimal tissue quality. This meant that any synaptic features, such as vesicles, pre and postsynaptic densities and the synaptic cleft would be clearly visible.
\section{Project outlook}
\label{sec:outlook}
Several possible experiments can be conceived to follow up on this project. Primarily, the low sample size should be increased considerably by imaging and reconstructing more VIP+ and in particular PV+ cells and adjacent cholinergic axons to verify the consistency of these data. Here, most attention should be spent on maximizing the resolution and quality of the EM images to be able to accurately reconstruct and identify the small synapse-like structures and their morphology. Using a focussed ion beam SEM (FIBSEM) instead of the 3View system applied in this project might for example yield better ultrastructural data due to the intrinsically higher resolution of the system. Furthermore, cryofixation instead of chemical fixation might limit shrinkage of the ECS, which would be beneficial for investigating the natural nanostructures of interneuron contacts (see ch. \ref{sec:ECS}).
However, already the available dataset can yield more data. Additional structures like synaptic vesicles could be reconstructed and analyzed. Postsynaptic structures of ChAT+ axon contacts could be traced back and categorized to get a more defined view of the targets of ACh. \\
Moreover, the somatostatin (SST) expressing interneuron class has not been analyzed in this project, although it is known to be involved in the local disinhibitory circuit where ChAT+, VIP+ and PV+ neurons modulate each other and the activity of principal cells (fig. \ref{fig:disinhib_circuit}). Imaging and reconstructing SST+ cells and their connections with ChAT+ axons will provide a more complete view of the connectivity and mode of action of ACh on cortical inhibitory interneurons.

\section{Conclusion}
The aim of this project was to characterize VIP+, PV+ cells and ChAT+ axons and their connections. We found that VIP+ and PV+ cells receive more excitatory than inhibitory input, although the inhibitory input on distal VIP+ dendrites is still substantial, corroborating the importance of cross-talk between inhibitory interneurons in cortical microcircuits. \\
Additionally, considering the connections made by neuromodulatory axons such as cholinergic axons, it may not be correct to impose strict rules on synapse characterization at the EM level. Despite analysing 333 \textmu m of ChAT+ axons we did not observe clear synaptic connections. However, vesicles were abundant, occasionally in close proximity to membrane contacts with interneurons. Therefore, it would be important to understand where the ACh receptors are located to fully identify the nature of the specificity of these contacts. The results indicate that a less stringent approach needs to be taken when analysing connections of neuromodulatory axons in EM images. This could be particularly relevant in the light of the significant investments that are being made into mapping neural connections of entire microcircuits to produce complete circuit diagrams of the brain. Refraining from clearly categorizing these contacts into synapse or non-synapses might enable a more neutral view on the mechanisms and functionality of the subtle and delicate communication between modulatory neurons and their targets.
